%%! program = pdflatex
%
%\documentclass[12pt,a4paper]{article} 
%\usepackage{arc-dp}
%\usepackage{amsfonts}
%\usepackage{amsmath}
%\usepackage{graphicx}
%\usepackage{times}
%\usepackage{setspace}
%\usepackage{fancyhdr}
%\usepackage{color}
%\usepackage[normalem]{ulem}
%\usepackage{subfig}
%\usepackage{float}
%\usepackage{caption}
%\usepackage{array}
%\usepackage{pgfgantt}
%\usepackage{url}
%\usepackage{enumitem}
%\usepackage{subfig}
%\usepackage{pgfgantt}
%\usepackage{wrapfig}
%\usepackage{enumitem}
%
%\usepackage{booktabs} % for spacing tables
%\usepackage{tabularx} % auto table sizing
%\usepackage{multirow} % table multirow
%\usepackage{soul}
%
%%\usepackage{epsf}
%%\usepackage{fancyheadings}
%%\usepackage{subfigure}
%%\usepackage{pst-gantt}
%%\usepackage{tweaklist}
%
%\newcolumntype{L}[1]{>{\raggedright\let\newline\\\arraybackslash\hspace{0pt}}m{#1}}
%\newcolumntype{C}[1]{>{\centering\let\newline\\\arraybackslash\hspace{0pt}}m{#1}}
%\newcolumntype{R}[1]{>{\raggedleft\let\newline\\\arraybackslash\hspace{0pt}}m{#1}}
%
%\let\OLDthebibliography\thebibliography
%\renewcommand\thebibliography[1]{
%  \OLDthebibliography{#1}
%  \setlength{\parskip}{1pt}
%  \setlength{\itemsep}{1pt plus 0.3ex}
%}
%
%
%%\renewcommand{\enumhook}{\setlength{\topsep}{0pt}%
% % \setlength{\itemsep}{-2mm}}
%%\renewcommand{\itemhook}{\setlength{\topsep}{0pt}%
%%  \setlength{\itemsep}{-2mm}}
%  %%%%%UNCOMMENT THE NEXT COMMAND IF NEEDED
%%\renewcommand{\descripthook}{\setlength{\topsep}{0pt}%
%%  \setlength{\itemsep}{-2mm}}
%
%%\pagestyle{fancy}
%
%%\input{psfig.sty}
%\newcommand{\todo}[1]{\textcolor{red}{#1}}
%\newcommand{\rules}[1]{\textcolor{blue}{#1}}
%\newcommand{\pset}{ {\rm P} \! \! \! {\rm P} }
%\date{}
%%\include{psfig}
%\remove{
%\topmargin -15mm
%\headheight 0pt
%\headsep 0pt
%\textheight 285mm
%\oddsidemargin -15mm
%\evensidemargin -15mm
%\textwidth 190mm
%\columnsep 10mm
%\marginparwidth 0pt
%\marginparsep 0pt
%}
%
%\usepackage[top=0.5cm, bottom=0.5cm, left=0.5cm, right=0.5cm]{geometry}
%\parindent=4mm
%\parskip=0.2mm
%
%%\usepackage{geometry} % see geometry.pdf on how to lay out the page. There's lots.
%%\geometry{a4paper} % or letter or a5paper or ... etc
%% \geometry{landscape} % rotated page geometry
%
%
%%\linespread{1.5}
%
%\newcommand*{\TitleFont}{%
%      \usefont{\encodingdefault}{\rmdefault}{b}{n}%
%      \fontsize{12}{12}%
%      \selectfont}
%
%\title{A unified accessible platform for integration of urban analytics and human thermal comfort modelling}
%%\author{}
%\date{} % delete this line to display the current date
%
%%%% BEGIN DOCUMENT
%\begin{document}
%\rmfamily
%\date{}


\noindent \textbf{F17-ROPE-Details of the participant's academic career and opportunities for research, evidence of research impact and contributions to the field, including those most relevant to this application }\\ \noindent 





\subsection*{\TitleFont Amount of Time as an Active Researcher}

I was awarded my PhD on urban micro-climate modelling from the School of Earth, Atmosphere and Environment at Monash University in March 2017. I have been an active researcher (at 1.0 FTE) for 3 years and 7 months post-PhD without interruption.

\subsection*{\TitleFont Research Opportunities}

\textbf{Past research opportunities}

My research career has greatly benefited from the skills and experience I developed throughout my previous career in software engineering. This previous career involved 13 years as a senior level software engineer, at a number of companies, but most significantly 8 years at LexisNexis/Reed Elsevier. This career developed my ability to break down complex problems into components and algorithms and fit these pieces back together into efficient process flows. It required designing interfaces to allow users to control complex business process flows as well as the components on the backend to perform the business logic and store the results. It has also given me a high level of expertise in software development and design and proficiency in a number of computer languages (including C++, Python, XML, FORTRAN, and especially Java). I also gained years of experience in project development and management (especially using the Agile methodology), guiding multi-year projects, developed by resources located around the world, and delivered to multiple business sites across the global company locations. These project managements skills have proven valuable in academic research in structuring large research programs, working together with remote collaborators and colleagues, and while supervising students and helping them understand how to conduct research.

Following my industry career, I returned to university and completed a Master's degree in Environment and Sustainability through the School of Geography at Monash University. My final semester research project was an observational and modelling study examining the micro-climate of mixed urban and parkland environments. As a result of this project, I was engaged by the CRC for Water Sensitive Cities to write a report recommending a suitable micro-climate model to model the thermal comfort impacts of water sensitive urban design (especially urban vegetation and water features). My PhD at Monash University then followed from this, having not found a suitable model in the report, set out to design and build such a model. VTUF-3D and later TARGET resulted from my PhD research and subsequent collaborations with other researchers at Monash.

\textbf{Current roles}

My post-doctoral career has been split (50/50) between two research fellow positions (funded separately) that encompass two different research areas. The first position is as a Research Fellow (research only and currently at 0.5 FTE)  with the Transport, Heath, and Urban Design (THUD) Research Lab in the Faculty of Architecture, Building, and Planning at the University of Melbourne. This involves research using innovative technologies (artificial intelligence, big data analysis, agent-based modelling, computer vision techniques as well as more traditional statistical methods) to examine multiple aspects of urban areas and the contribution of urban design on public health. The lab is a highly supportive and collaborative group of researchers, from which I have published 9 papers in 2019-2020 (6 in Q1 journals). This is a highly competitive amount of output in comparison to the field in general.

My second position is a 2 year 0.5 FTE research contract (subcontracted through the University of Melbourne) as a Research Fellow and urban climate modelling scientist with Monash University and the CRC for Water Sensitive Cities. My achievements, even at an early research career stage, have led to recognition as an expert in vegetation and human thermal comfort modelling at a micro-climate scale, leading to this external funding from the CRC for Water Sensitive Cities to further develop this work and consult with local and state government. This role is split between 20\% consulting, 40\% research, and 40\% tool development. The research portion has largely been devoted to assessments of urban heat outcomes for different urban infill development scenarios. I designed and performed an urban heat analysis of a number of different green field development scenarios in Sunbury for the CRC, especially considering water sensitive (and climate sensitive) urban design. This resulted in a report for the CRC, "Estimating the economic benefits of Urban Heat Island mitigation – Biophysical Aspects"\footnotemark \footnotetext{https://watersensitivecities.org.au/content/estimating-the-economic-benefits-of-urban-heat-island-mitigation-biophysical-aspects/}. The tool development is devoted to continued development of my urban climate models VTUF-3D and TARGET and their integration with the CRC's scenario planning support tool. I also used this opportunity to improve the performance and usability of my climate models, as very few models can quantify the human thermal benefits of urban green and blue space, especially accounting for cooling effects of vegetation and water evaporation. The consulting has included urban heat modelling for state and local government, often joint projects with consulting companies such as GHD. For example, projects resulted in contributions of urban heat assessments to the Urban Ecology Strategy for Fishermans Bend for the Victoria Department of Environment, Land, Water \& Planning (DELWP) and serving on the science panel for the development of the Cool Suburbs Tool for the Western Sydney Regional Organisation of Councils (WSROC). 

While these two positions sit in different disciplines, they overlap in the development and use of modelling and techniques to examine urban areas and the health impacts of urban design issues. In addition, my software development skills and computer vision technical knowledge has proved transferable to undertaking research in a wide range of disciplines including urban analytics, leading to new innovations in quantifying health impacts of urban design and transportation infrastructure. This has led to my role as a key contributor to the Transport, Heath, and Urban Design Research Lab (THUD). My initial task in THUD was to organise and write an ARC Linkage application, 'A Multi-criteria Design Platform to Facilitate Active School Journeys', quantifying topography, street network connectivity, traffic risk, pollution levels, and thermal comfort. I was not a named participant but the application was submitted in December 2017 (and resubmitted and funded in 2020). 

I co-developed the neural network clustering and analysis technique used in the lab's recent Lancet Planetary Health publication. This allowed the identification of city types from map segments from the 1700 largest global cities at higher risk of road trauma. This method was expanded in a more recent publication (with myself as first author) in Urban Science to also include street view imagery and satellite imagery to derive urban typologies. Also, in conjunction with other lab collaborators, I developed a method to identify neighbourhood typologies (`block typologies') using self organising maps to cluster metrics extracted from map segments. This study is currently in-review (with 600 reads on Researchgate). All three of these projects were used as the base methodology for our lab's recently awarded NHMRC/UKRI research grant. My contribution to the lab is also represented in many of the 9 journal publications developed my the lab in the last 3 years.

My expertise in urban heat modelling and computer vision techniques have led to me being sought out to participate as a principal investigator in submitted (in-review) grant applications including: ARC Discovery DP210102089 `Sustainable mobility: city-wide exposure modelling to advance bicycling' headed by Dr Ben Beck (Monash University) and NHMRC Ideas application 2002025 `Pathways to health: advancing bicycling as an active mode of transport' headed by Dr Ben Beck (Monash University). I am also a co-investigator on a (submitted) Swiss National Science Foundation grant application `Heat-Down: Integrated modelling of stormwater and urban heat for cooling cities' headed by Dr. Jo\~{a}o P. Leit\~{a}o (Eawag) and Dr. Peter M. Bach (ETH Zurich). These are in addition to awarded grants detailed in the section below.



Through my academic career, I have been fortunate to receive excellent mentoring and career guidance. My PhD was supervised by Prof. Nigel Tapper (Monash University; current president of The International Association for Urban Climate) and Dr. Andrew Coutts (Monash University, a leading urban climate researcher). Prof. Tapper remains a frequent collaborator and also supervises one of my current positions. At the University of Melbourne, Prof. Mark Stevenson (a professor of Urban Transport and Public Health and NHMRC Research Fellow) supervises my other position and along with other members of the lab (Dr. Jason Thompson particularly) provides valuable mentoring.

\subsection*{\TitleFont Research Achievements and Contributions}

\textbf{How my research has led to advances in knowledge in the field. How will my achievements contribute to the application:}


As an early-career researcher, I have quickly built a large body of work. These research achievements include three urban climate models able to examine urban heat mitigation strategies and make predictions of human thermal stress at a local and micro-scale. These models have been adopted by other researchers and consultants. My knowledge about modelling and model development has led to my being included in research projects and grant applications to further develop these models and contribute to the development of other models. In addition, methods that I have developed using computer vision techniques to cluster similar types of urban areas and examine the links of the design to public health outcomes and have formed the basis of successful grant applications and these techniques are currently being utilised to formulate appropriate public health measures in these areas. 


I have developed a strong collaborative network (within my universities, Australia, and internationally), and developed a strong research direction based on modelling and quantifying urban systems. This rapid upward trajectory has been strongly enabled by a previous long career in industry and software engineering that required the ability to develop and organise large projects, solve problems, and build the tools necessary to deliver results. This combination of long experience in computing techniques with deep urban climate and urban climate modelling development knowledge demonstrate that I am the ideal researcher to undertake and deliver this project.


The following papers and conference presentations are most strongly related to this DECRA application (i.e. climate model development, human thermal comfort modelling, and the collection of urban morphology information through databases and extraction from urban imagery).

\begin{list}{}{}
\itemsep-0.5em
\item [1.] \textbf{Nice, K. A.}, Coutts, A., and Tapper, N.J., Development of the VTUF-3D v1.0 urban micro-climate model to support assessment of urban vegetation influences on human thermal comfort. \textit{Urban Climate}, 2018. \\ doi:10.1016/j.uclim.2017.12.008.
\item [2.] Broadbent, A., Coutts, A., \textbf{Nice, K.}, Demuzere, M., Krayenhoff, E., Tapper, N. and Wouters, H., The Air-temperature Response to Green/blue-infrastructure Evaluation Tool (TARGET v1.0): an efficient and user-friendly model of city cooling. \textit{Geosci. Model Dev.}, 2019. 
\item [3.] Naika Meili, Gabriele Manoli, Paolo Burlando, Elie Bou-Zeid, Winston T.L. Chow, Andrew M. Coutts, Edoardo Daly, \textbf{Kerry A. Nice}, Matthias Roth, Nigel J. Tapper, Erik Velasco, Enrique R. Vivoni, and Simone Fatichi, An urban ecohydrological model to quantify the effect of vegetation on urban climate and hydrology (UT\&C v1.0), \textit{Geosci. Model Dev.}, 2020. 
\item [4.] Dommenget, D., \textbf{Nice, K.}, Bayr, T., Kasang, D., Stassen, C., and Rezny, M.: The Monash Simple Climate Model Experiments (MSCM-DB v1.0): An interactive database of mean climate, climate change and scenario simulations, \textit{Geosci. Model Dev.}, 2019. doi:10.5194/gmd-12-2155-2019
\item [5.] G\'{a}l, C. V and \textbf{Nice, K. A.} ‘Mean radiant temperature modeling outdoors: A comparison of three approaches’, in \textit{100th Annual Meeting of the American Meteorological Society (AMS) jointly with the 15th Symposium on the Urban Environment}, 2020. 
\item [6.] Todorovic, Tatjana, London, Geoffrey, Bertram, Nigel, Sainsbury, Oscar, Renouf, Marguerite A, \textbf{Nice, Kerry A} and Kenway, Steven J. 2019. ‘Models for water sensitive middle suburban infill development’, in\textit{ 9th State of Australian Cities National Conference, 30 November - 5 December 2019, Perth, Western Australia}. doi: 10.25916/5efa774bda643. 
\item [7.] \textbf{Nice, K.A.}, Wijnands, J. S., Middel, A., Wang, J., Qiu, Y., Zhao, N., Thompson, J., Aschwanden, G.D.P.A., Zhao, H., and Stevenson, M., Sky pixel detection in outdoor imagery using an adaptive algorithm and machine learning, \textit{Urban Climate}, 2020. 
\item [8.] \textbf{Nice, K. A.}, Targeted urban heat mitigation strategies using urban morphology databases and micro-climate modelling to examine the urban heat profile. \textit{EGU General Assembly 2020}.

\end{list}

Further other research outputs support my expertise in urban climates, urban typology clustering and urban analytics.
\begin{list}{}{}
\itemsep-0.5em
\item [9.] \textbf{Nice, K. A.}, Climate science context around urban cooling. In: \textit{4th Water Sensitive Cities Conference 2019, Brisbane}. \textbf{Invited talk.}
\item [10.] \textbf{Nice, K. A.}, Urban Greening for improved human thermal comfort. In: \textit{202020 Vision, The Green Light Tour, 2018, Adelaide}. \textbf{Invited talk.}
\item [11.] \textbf{Nice, K. A.}, Designing liveable cities through heat mitigation: tools to translate knowledge into design. In: \textit{3rd Water Sensitive Cities Conference, 2017, Perth.} \textbf{Invited talk.}
\item [12.] Thompson, J., Stevenson, M., Wijnands, J. S., \textbf{Nice, K.}, Aschwanden, G.D.P.A., Silver, J., Nieuwenhuijsen, M., Rayner, P., Schofield, R., Hariharan, R., and Morrison, C. N., A global analysis of urban design types and road transport injury: an image processing study, \textit{The Lancet Planetary Health}, 2020, doi:10.1016/S2542-5196(19)30263-3. 
\item [13.] \textbf{Nice, K.A.}, Thompson, J., Wijnands, J. S., Aschwanden, G.D.P.A, Stevenson, M., The “Paris-end” of town? Deriving urban typologies using three imagery types, \textit{Urban Sci.}, 2020.
\item [14.] Wijnands, J., \textbf{Nice, K.}, Thompson, J., Zhao, H. and Stevenson, M. Streetscape augmentation using generative adversarial networks: optimising health and wellbeing., \textit{Sustainable Cities and Society}, 2019.
\item [15.] \textbf{Nice, K.A.}, The Nature of Human Settlement: Building an understanding of high performance city design. In: \textit{UrbanSys2019/2019 Conference on Complex Systems, Singapore.}
\end{list}


\textbf{Invited keynote and speaker addresses:}

I have presented my research at 8 international and 7 (3 invited) national conferences since 2012. Three of these have been at the International Conference on Urban Climate (ICUC), the leading conference for urban climate. Two of these ICUC talks were about my VTUF-3D model and one was about using computer vision techniques and Google Street View to discover urban morphology parameters (namely sky view factors). Two recent international conference presentations have been on the topic of deriving urban typologies through big data urban imagery datasets and machine learning and computer vision. I have given three invited presentations on the topics of urban heat and designing heat mitigation strategies at the 3rd and 4th Water Sensitive Cities Conferences in Perth and Brisbane and at a 202020 Vision Green Light Tour in Adelaide. I have also been invited to present five guest lectures at Monash University and the University of Melbourne on the topic of urban climate modelling and the urban heat benefits of water sensitive urban design. 

\textbf{Research income:}

ADD: Chief Investigator on 2021-2023 \$422,000 ARC Discovery DP210102089, Sustainable mobility: city-wide exposure modelling to advance bicycling, Dr Ben Beck, Monash University

ADD: Partner Investigator on 2021, Swiss National Science Foundation, Project ID 200021\_201029, 453,764 CHF, Heat-Down: Integrated modelling of stormwater and urban heat for cooling cities

I have secured AUD \$785,910 and GBP \textsterling479,387 in research income through competitive grants over the last four years.

In 2016 I was awarded the \$10,000 Graham Treloar Early Career Researcher Fellowship (The University of Melbourne Faculty of Architecture, Building and Planning) for the development of the project `Urban canyon mean radiant temperatures predictions through mining Google Street View imagery and neural network machine learning'.

I am an investigator on the AUD \$608,910 (and GBP \textsterling479,387) 2020-2023 UKRI/NHMRC grant 1194959, `A Vision of Healthy Urban Design for NCD Prevention'. The methodology for this grant utilises neural networks and computer vision techniques to process large amounts of urban imagery to assess the impacts of urban design on non-communicable disease (NCD). This is a collaborative project between researchers at the University of Melbourne and Queen's University Belfast.

I secured a \$137,000 research contract with the CRC for Water Sensitive Cities as a specialist cohort in urban heat modelling. This contract was solicited by the CRC to provide urban heat expertise to the final two years of the CRC research program and provides two years of 0.5 FTE funding over 2019-2020.

I am a principal investigator on a 2020 AUD\$30,000 Melbourne Energy Institute grant on `The effects of COVID-19 on reduced transport and emissions for global city typologies', from which the paper 'The impact of the COVID-19 pandemic on air pollution: A global assessment using machine learning techniques' was recently submitted to Atmospheric Pollution Research.

\textbf{Research supervision, mentoring and advice:}

I am currently supervising two PhD students. One is at Monash University observing and modelling the cooling potential of irrigation of the runway buffer areas at Adelaide Airport. The other at the University of Melbourne is looking at the cooling potential and energy balances of misting systems and other active irrigation techniques in outdoor areas. I have also supervised the final capstone research projects for 11 Masters of IT (MIT) students at the University of Melbourne. Methods from 3 of these MIT projects were incorporated into publication \#7 listed above. In addition, methods from 1 other MIT project is currently being incorporated into a health/computer vision collaboration with researchers from Cambridge University. Finally, I have been invited to participate in 3 PhD review panels for the urban climate discipline.



\textbf{Benefits outside academia:}

My expertise in urban heat modelling has been utilised in a number of government consultations and planning reports in 2019-2020. These include serving on the science advisory panel for Western Sydney Regional Organisation of Councils (WSROC) Cool Suburbs Rating and Accreditation tool, providing modelling and urban heat analysis for the Queensland Department of Environment and Science (DES), reports for urban heat impacts of infill development for South Australia (Salisbury, an Adelaide suburb) and Western Australia (the Perth suburb of Kutsford), and finally project work for the ACT government's micro-climate urban heat strategies.



\textbf{Other professional activities:}

I maintain memberships in the European Geosciences Union (EGU), The Australian Meteorological and Oceanographic Society (AMOS), The International Association for Urban Climate (ICUC), and the Complex Systems Society (CSS).

Article Referee Activities: In the past 4 years, I have performed 30 peer reviews for 13 leading climate and urban design journals, including Urban Climate, Theoretical and Applied Climatology, Atmosphere, Sustainable Cities and Society, Environment and Planning B, and Building Simulation.





\subsection*{\TitleFont F18. Research Opportunity and Performance Evidence (ROPE) - Research Output Context - Research context: }

\textbf{Provide clear information that explains the relative importance of different research outputs and expectations in the participant's discipline/s. The information should help assessors understand the context of the participant's academic research achievements but not repeat information already provided in this application. It is helpful to include the importance/esteem of specific journals in their field; specific indicators of recognition within their field such as first authorship / citations, or significance of non-traditional research outputs. (Up to 3,750 characters, approximately 500 words).)}

I currently have 13 journal articles (3 as first author and 3 as second author) and 2 refereed full-length conference papers. I have presented my research at 8 international conferences and 7 national conferences with 3 of those as invited talks. My research has also been featured in 5 collaborative presentations at 5 international conferences (1 as an invited talk). I have 57 citations listed in Google Scholar and a h-index of 4. I have 5 reports written for the CRC for Water Sensitive Cities and 1 book chapter. My publication output compares favourably (as shown in an Elsevier SciVal analysis detailed in Section D1) to a selection of other respected urban climate researchers despite all my research only having been published starting in 2018. 

The first stage of my academic career has been in urban climate modelling, the topic of my PhD. The model I developed in my PhD, VTUF-3D, was published (publication \#1 of my career top 10) in Urban Climate (SJR Q1 1.042, the central journal for the urban climate discipline) with my supervisors as co-authors. To date, this is my most cited work. It remains one of the few models able to assess the cooling impacts of urban vegetation at a micro-scale, and has led to further collaborations and model development. It has also had a large engagement 576 reads on Researchgate, while my PhD thesis, from which this article was developed, has had 1,305 reads. 

Three additional climate models, TARGET, UT\&C, and MSCM-DB, co-developed through collaborations, were all published (\#6, \#9, and MSCM-DB) in Geoscientific Model Development (SJR Q1 3.18, a principal journal for geophysical modelling development). Publication \#8 explores the state of the art (including my VTUF-3D model) in modelling outdoor mean radiant temperatures (a key parameter for predicting heat stress) published in conjunction with the American Meteorology Society annual conference. Wider multidisciplinary applications of (and a framework around) water usage and urban design on both urban heat and other issues of urban liveability were developed for the State of Australian Cities conference (\#10). I have utilised computer vision techniques in support of urban climate modelling. This led to a method to more accurately detect sky pixels (a preliminary step in calculating sky view factors) using a range of urban imagery types, an invited first author publication in Urban Climate (\#4). 

My research also expanded to include artificial intelligence to examine urban systems. Publication \#2, published in The Lancet Planetary Health (SJR Q1 4.205), used neural networks to cluster the largest 1700 global cities using millions of maps and examine the impact of urban design types on road trauma. Additional imagery types (satellite and street view) were added to construct urban typologies (\#3, published in Urban Science). Other types of artificial intelligence, generative adversarial networks were used to transform street level and satellite imagery from areas with poor health outcomes into new imagery, providing insights into the urban factors leading to these poor outcomes (\#7 in Sustainable Cities and Society, SJR Q1 1.356). A deep autoencoder extracted features from satellite imagery of all the intersections in Australia to identify safe intersection design, \#5 published in Computer-Aided Civil and Infrastructure Engineering (SJR Q1 1.874).

Many of these papers have attracted attention. Four papers received an Altmetric Attention Score of 10 or higher. Seven are over 5. The Lancet Planetary Health paper has a score over 169, including coverage by four different news outlets and the Urban Science paper has reached 17.


%\end{document}
