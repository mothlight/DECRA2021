%%! program = pdflatex
%
%\documentclass[12pt,a4paper]{article} 
%\usepackage{arc-dp}
%\usepackage{amsfonts}
%\usepackage{amsmath}
%\usepackage{graphicx}
%\usepackage{times}
%\usepackage{setspace}
%\usepackage{fancyhdr}
%\usepackage{color}
%\usepackage[normalem]{ulem}
%\usepackage{subfig}
%\usepackage{float}
%\usepackage{caption}
%\usepackage{array}
%\usepackage{pgfgantt}
%\usepackage{url}
%\usepackage{enumitem}
%\usepackage{subfig}
%\usepackage{pgfgantt}
%\usepackage{wrapfig}
%\usepackage{enumitem}
%
%\usepackage{booktabs} % for spacing tables
%\usepackage{tabularx} % auto table sizing
%\usepackage{multirow} % table multirow
%\usepackage{soul}
%
%%\usepackage{epsf}
%%\usepackage{fancyheadings}
%%\usepackage{subfigure}
%%\usepackage{pst-gantt}
%%\usepackage{tweaklist}
%
%\newcolumntype{L}[1]{>{\raggedright\let\newline\\\arraybackslash\hspace{0pt}}m{#1}}
%\newcolumntype{C}[1]{>{\centering\let\newline\\\arraybackslash\hspace{0pt}}m{#1}}
%\newcolumntype{R}[1]{>{\raggedleft\let\newline\\\arraybackslash\hspace{0pt}}m{#1}}
%
%\let\OLDthebibliography\thebibliography
%\renewcommand\thebibliography[1]{
%  \OLDthebibliography{#1}
%  \setlength{\parskip}{1pt}
%  \setlength{\itemsep}{1pt plus 0.3ex}
%}
%
%
%%\renewcommand{\enumhook}{\setlength{\topsep}{0pt}%
% % \setlength{\itemsep}{-2mm}}
%%\renewcommand{\itemhook}{\setlength{\topsep}{0pt}%
%%  \setlength{\itemsep}{-2mm}}
%  %%%%%UNCOMMENT THE NEXT COMMAND IF NEEDED
%%\renewcommand{\descripthook}{\setlength{\topsep}{0pt}%
%%  \setlength{\itemsep}{-2mm}}
%
%%\pagestyle{fancy}
%
%%\input{psfig.sty}
%\newcommand{\todo}[1]{\textcolor{red}{#1}}
%\newcommand{\rules}[1]{\textcolor{blue}{#1}}
%\newcommand{\pset}{ {\rm P} \! \! \! {\rm P} }
%\date{}
%%\include{psfig}
%\remove{
%\topmargin -15mm
%\headheight 0pt
%\headsep 0pt
%\textheight 285mm
%\oddsidemargin -15mm
%\evensidemargin -15mm
%\textwidth 190mm
%\columnsep 10mm
%\marginparwidth 0pt
%\marginparsep 0pt
%}
%
%\usepackage[top=0.5cm, bottom=0.5cm, left=0.5cm, right=0.5cm]{geometry}
%\parindent=4mm
%\parskip=0.2mm
%
%%\usepackage{geometry} % see geometry.pdf on how to lay out the page. There's lots.
%%\geometry{a4paper} % or letter or a5paper or ... etc
%% \geometry{landscape} % rotated page geometry
%
%
%%\linespread{1.5}
%
%\newcommand*{\TitleFont}{%
%      \usefont{\encodingdefault}{\rmdefault}{b}{n}%
%      \fontsize{12}{12}%
%      \selectfont}
%
%\title{Maximising cooling through blue-green infrastructure enabled urban planning}
%%\author{}
%\date{} % delete this line to display the current date
%
%%%% BEGIN DOCUMENT
%\begin{document}
%\rmfamily
%\date{}
%
% 


\subsection*{\TitleFont A1. Title: (75 characters) }


Achieving urban heat mitigation through blue-green infrastructure



\subsection*{\TitleFont A4. Application summary: (750 characters/100 words) }
%Summary focused on aims, significance, expected outcomes and benefits of the project. Clear plain English.

%Audience is detailed assessor (discipline and subject matter experts) and the College of Experts panel





%This project aims to protect urban areas from extreme heat, the most dangerous Australian natural hazard, through the cooling benefits of blue-green infrastructure (vegetation and water features). This is significant because despite the evidence that the use of these features can be an effective method of urban cooling, we lack detailed observations to understand how to maximise the benefits of their inclusions in urban designs and to build the modelling tools needed for systematic scenario testing. This project provide these observations and use them to build the models needed to systematically test scenarios and find the most optimal methods of urban heat mitigation and adaptation strategies. The benefits will be more heat resilient urban areas and identification of areas of high vulnerability requiring immediate attention.

Extreme heat is Australia’s most dangerous natural hazard. This project aims to protect urban areas from extreme heat through the cooling benefits of blue-green infrastructure including vegetation and water features. This is significant because despite evidence that use of these features can be an effective method of urban cooling, we lack detailed observations to understand how to optimise their benefits in urban design. This project aims to provide these observations and use them to build models needed to systematically test scenarios and find optimal urban heat mitigation and adaptation strategies. The benefits will be more heat resilient urban areas and identification of areas of high vulnerability that can benefit from interventions.


\subsection*{\TitleFont A5. List objectives of proposed project: (500 characters/70 words per objective) }



%Words words words words words words words words words words words words words words words words words words words words words words words words words words words words words words words words words words words words words words words words words words words words words words words words words words words words words words words words words words words words words words words words words words words words words words words words words words words words words words words words words words words.


To collect observations needed to understand the underlying processes driving the cooling benefits of a wide range of blue-green infrastructure features.

To upgrade urban climate modelling tools and show them suitable to model human thermal comfort benefits of blue-green infrastructure and urban water usage.

To consult with key stakeholders, especially water companies, and co-design this research project, mapping out the tools and results that will be of highest value to enable their design processes to maximise the human thermal comfort benefits of using blue-green infrastructure in urban areas.

Quantify blue-green infrastructure cooling impacts through systematic modelling to discover optimal scenario parameters, design limitations, and ranked priorities to guide redesign efforts to mitigate urban heat.


\subsection*{\TitleFont A6. National Interest Test Statement: (750 to 1125 characters/100 to 150 words) }
%How will research contribute to Australia's national interest through economic, commercial, social or cultural benefits to Australian community.

%Urban heat has large impacts on public health with risk disproportionately borne by the elderly and the very young. Strategies, especially those that utilise blue-green infrastructure such as increased vegetation cover, water features are available to mitigate these problems. This project will develop urban modelling tools needed for academic researchers in city science and urban climate to examine strategies to mitigate urban heat with blue-green infrastructure. In addition, the platform built around these tools will bring these sophisticated tools to practitioners who need to make immediate decisions about the future design of cities and allow assessments to be made about urban heat mitigation and adaptation strategies using vegetation and the use of water practices. The analysis tools can also be used to examine urban areas for hotspots, areas of high vulnerability, that require immediate attention for remediation to reduce the vulnerability and to provide warning to emergency responders and crisis services for areas that might require extra attention during heat waves.


%Increase capacity to address urban heat and alleviates heat health risks
%Provides a better understanding of how much cooling can be provided by BGI and how to maximise its impact.

 
 
This project will provide benefits to the Australian community by increasing Australia's capacity to respond to environmental change, through more resilient urban, rural and regional infrastructure and providing improved options for urban areas to adapt to climate impacts. The project will increase our understanding of how blue-green infrastructure, such as increased vegetation cover and water features, can provide cooling benefits to urban areas and how to best utilise these features to maximise their cooling potentials. It will also increase the capacity of Australian urban planners and policy makers to mitigate urban heat impacts, alleviate the associated health risks, and reduce the burdens on Australia's health systems. This research program will be co-designed in conjunction with industry stakeholders, especially water companies and property developers, to provide them with the knowledge and tools to make decisions about the future design of cities as well as the retrofitting of existing areas to reduce heat vulnerability for all those living in those areas.



 

%. These benefits will be realised predominately through the increased capability for urban planners to mitigate the risks of urban heat through resilient urban, rural and regional infrastructure. 

%provide a better understanding of how blue-green infrastructure, such as increased vegetation cover and water features, can provide cooling benefits to urban areas and how to best utilise these features to maximise their cooling potentials.

%mitigate urban heat impacts, alleviate the associated health risks, and reduce the burdens on Australia's health systems.


%Outline the extent to which the research contributes to Australia’s national interest through its potential to have economic, commercial, environmental, social or cultural benefits to the Australian community. Write the description of the national interest simply, clearly and in plain English between 750 and 1125 characters (between approximately 100 and 150 words).
%Note: The National Interest Test Statement may also be publicly released by the ARC. 



%Application specific comments:
%
%Outline the extent to which the research contributes to Australia’s national interest through it’s potential to have economic, commercial, environmental, social or cultural benefits to the Australian community. This statement should emphasise the ‘why’ rather than repeat the ‘what’
%
%Suitable NIT Statements relate proposed research to policies or government initiatives, or to industry and economic values. 
%
%Be clear about who benefits from the research—all, or a specific section of Australia. 
%
%Some directions include (amend your statement as appropriate):
%
%  Development of a new product, process, industry or market (which then has a described value, savings or worth)
%
%  Relating the work to existing or proposed policies and the issues they are focussed on addressing – perhaps in reports, commissions, data
%
%  Applications of the work, which then have a benefit
%
%  Increased understanding of something – which then has a described benefit that ensues
%
%  Better capacity to address a current problem – with what this alleviates or solves described.
%
%
%
%GENERAL comment for all applicants: This statement may also be used for public release by the ARC.





%
%\end{document}
