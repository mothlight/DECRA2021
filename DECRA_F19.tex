%


%! program = pdflatex
%
%\documentclass[12pt,a4paper]{article} 
%\usepackage{arc-dp}
%\usepackage{amsfonts}
%\usepackage{amsmath}
%\usepackage{graphicx}
%\usepackage{times}
%\usepackage{setspace}
%\usepackage{fancyhdr}
%\usepackage{color}
%\usepackage[normalem]{ulem}
%\usepackage{subfig}
%\usepackage{float}
%\usepackage{caption}
%\usepackage{array}
%\usepackage{pgfgantt}
%\usepackage{url}
%\usepackage{enumitem}
%\usepackage{subfig}
%\usepackage{pgfgantt}
%\usepackage{wrapfig}
%\usepackage{enumitem}
%
%\usepackage{booktabs} % for spacing tables
%\usepackage{tabularx} % auto table sizing
%\usepackage{multirow} % table multirow
%\usepackage{soul}
%
%%\usepackage{epsf}
%%\usepackage{fancyheadings}
%%\usepackage{subfigure}
%%\usepackage{pst-gantt}
%%\usepackage{tweaklist}
%
%\newcolumntype{L}[1]{>{\raggedright\let\newline\\\arraybackslash\hspace{0pt}}m{#1}}
%\newcolumntype{C}[1]{>{\centering\let\newline\\\arraybackslash\hspace{0pt}}m{#1}}
%\newcolumntype{R}[1]{>{\raggedleft\let\newline\\\arraybackslash\hspace{0pt}}m{#1}}
%
%\let\OLDthebibliography\thebibliography
%\renewcommand\thebibliography[1]{
%  \OLDthebibliography{#1}
%  \setlength{\parskip}{1pt}
%  \setlength{\itemsep}{1pt plus 0.3ex}
%}
%
%
%%\renewcommand{\enumhook}{\setlength{\topsep}{0pt}%
% % \setlength{\itemsep}{-2mm}}
%%\renewcommand{\itemhook}{\setlength{\topsep}{0pt}%
%%  \setlength{\itemsep}{-2mm}}
%  %%%%%UNCOMMENT THE NEXT COMMAND IF NEEDED
%%\renewcommand{\descripthook}{\setlength{\topsep}{0pt}%
%%  \setlength{\itemsep}{-2mm}}
%
%%\pagestyle{fancy}
%
%%\input{psfig.sty}
%\newcommand{\todo}[1]{\textcolor{red}{#1}}
%\newcommand{\rules}[1]{\textcolor{blue}{#1}}
%\newcommand{\pset}{ {\rm P} \! \! \! {\rm P} }
%\date{}
%%\include{psfig}
%\remove{
%\topmargin -15mm
%\headheight 0pt
%\headsep 0pt
%\textheight 285mm
%\oddsidemargin -15mm
%\evensidemargin -15mm
%\textwidth 190mm
%\columnsep 10mm
%\marginparwidth 0pt
%\marginparsep 0pt
%}
%
%\usepackage[top=0.5cm, bottom=0.5cm, left=0.5cm, right=0.5cm]{geometry}
%\parindent=4mm
%\parskip=0.2mm
%
%%\usepackage{geometry} % see geometry.pdf on how to lay out the page. There's lots.
%%\geometry{a4paper} % or letter or a5paper or ... etc
%% \geometry{landscape} % rotated page geometry
%
%
%%\linespread{1.5}
%
%\newcommand*{\TitleFont}{%
%      \usefont{\encodingdefault}{\rmdefault}{b}{n}%
%      \fontsize{12}{12}%
%      \selectfont}
%
%\title{A unified accessible platform for integration of urban analytics and human thermal comfort modelling}
%%\author{}
%\date{} % delete this line to display the current date
%
%%%% BEGIN DOCUMENT
%\begin{document}
%\rmfamily
%\date{}

\subsection*{\TitleFont F19. Research Opportunity and Performance Evidence (ROPE) - Research Output Context - Research context: }

\textbf{Provide clear information that explains the relative importance of different research outputs and expectations in the participant's discipline/s. The information should help assessors understand the context of the participant's academic research achievements but not repeat information already provided in this application. It is helpful to include the importance/esteem of specific journals in their field; specific indicators of recognition within their field such as first authorship / citations, or significance of non-traditional research outputs. (Up to 3,750 characters, approximately 500 words).)}

I currently have published 14 journal articles (3 as first author and 3 as second author) and 3 refereed full-length conference papers. In the four years that I have been publishing (since 2018), I have already accumulated 181 citations in Google Scholar and a h-index of 7. My citations are increasing rapidly, with 90 of them occurring in 2021. The article on my VTUF-3D model is my most cited work. I have 7 reports written for the CRC for Water Sensitive Cities and 1 book chapter. My publication output compares favourably (as shown in an Elsevier SciVal analysis detailed in Section F18) to a selection of other highly respected urban climate researchers at similar or later career stages. 

My research crosses a number of different disciplines, urban climates, climate modelling, artificial intelligence, computer vision, urban analytics, urban design, and public health. For all the fields I have published in, authorship conventions are similar. Authorship is generally ordered by contribution level, with the first author leading the effort, the second and third authors generally also making large contributions, and the final author supervising the effort. Sole authorship is rare because of the collaborations and inter-disciplinary research required in these areas.

There are a wide variety of climate journals but Urban Climate (SJR Q1 1.151) has become the central journal for this discipline since 2012. Publications on climate modelling can be found across all these journals but Geoscientific Model Development (SJR Q1 3.238) is a leading journal modelling and simulation, especially geophysical modelling development. Author lists can include 5-10 authors as model development is an incremental process and next generation models generally build on the work of previous models.

The computer vision and artificial intelligence fields generally publish predominately in conference proceedings, but applied research using these techniques are more often published in domain specific journals. In my case, as my other research outside of urban climate field generally crosses multiple disciplines, I have published in interdisciplinary orientated journals. The Lancet family of journals are top tier in medicine and health, and the Lancet Planetary Health (SJR Q1 4.205) is a highly ranked interdisciplinary journal covering global health issues. Sustainable Cities and Society (SJR Q1 1.65) focuses on multi-disciplinary research into designing resilient cities. The Q1 journal Environment and Planning B focuses on state of the art analytical methods for urban planning and design. Computer-Aided Civil and Infrastructure Engineering (Q1 SJR 2.77) focuses on the the use of computer science in aid of engineering.

Many of these papers have attracted attention. Five papers received an Altmetric Attention Score of 10 or higher. Seven are over 5. The Lancet Planetary Health paper has a score of 165, including coverage by four different news outlets, the streetscape augmentation paper has reached 20, and the Urban Science paper 17.

Conferences in my various disciplines are either a combination of abstract submission and presentation or abstract submission, presentation, and then a fully peer-reviewed article in the proceedings. I have participated in both types. A peer-reviewed article resulted from three, the American Meteorological Society, the State of Australian Cities, and Digital Image Computing conferences.

Other research outputs from my career are made up of modelling code (the four models I have developed or co-developed) distributed through online public repositories. DOI numbers can be assigned to attract citations when the code is used by others, but in practice the code is rarely given recognition on its own and are only cited in other's academic work via the publications describing their development. Usage or adoption by consultants or other non-academic users will generally receive no public recognition.















%\end{document}
